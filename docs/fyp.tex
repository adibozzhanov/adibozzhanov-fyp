
%%% Local Variables:
%%% mode: latex
%%% TeX-master: t
%%% End:

\documentclass[12pt]{article}
\usepackage[utf8]{inputenc}
\usepackage{hyperref}
\usepackage[margin=2cm]{geometry}

\title{Google Scholars Data Visualisation}
\author{Adi Bozzhanov}

\begin{document}
\maketitle

\section{Introduction}

The aim of this document is to brainstorm about my Final Year Project topic
and try to structure my initial thoughts about it. The vague initial problem statement is as follows:

\textbf{"Parse data from google scholars and visualise it, potentially as a network of articles"}.

As a starting point, I will try to answer following questions before going into implementation
and planning details:


\begin{itemize}
\item What kind of data am I working with?
\item What are potential ways of visualising this data?
\item What is the potential development Stack?
\item What do I expect the final result to look like?
\end{itemize}

\section{What am I working with?}

This section is to just summarise what kind of information I can even get from google scholar.
The google scholar landing page is just a simple search bar where one can enter any prompt
and get a list of relevant scholarly articles as a result. Each article has the following
information that will be useful for network visualisation.

\begin{itemize}
\item Authors [Article - Author]
\item Cited by [Article - Article]
\item Refers to [Article - Article]
\item Number of citations
\end{itemize}

I believe this is everything that could be useful???
\section{What can I visualise?}

I could come up with those few useful ways to visualise the following data.
\begin{itemize}
\item Authors network [nodes: Authors, edges: articles, undirected graph]
\item Citations networks [nodes: Articles, edges: citations/references, directed graph]
\end{itemize}

\section{Potential Stack}

Considering that there already is a very nice \href{https://scholarly.readthedocs.io/en/latest/index.html}{scholarly} python library
for accessing the google scholar in a pretty easy fashion, I would like to implement the
majority of the project in python. Since it very likely will be a web application,
Django + React could be just the right combination for the job.

\section{Expected result}

So far the plan is fairly simple: Make a web application with a python
back-end which will provide the following features at the very least.
\begin{itemize}
\item Authors network visualisation and traversal
\item Publications network visualisation and traversal
\item Provide multiple ways of searching for a publication and rendering it's network
  (Integrating default google scholar search results generation)
\item Data filtering, sorting and basically search configuration tools.
\end{itemize}

The plan from this point would be to convert these thoughts into a technical specification,
mock ups, the architecture plan,
and come up with a development road map for the rest of the year.

\end{document}